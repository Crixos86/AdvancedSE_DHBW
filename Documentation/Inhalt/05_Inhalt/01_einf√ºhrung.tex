\chapter{Einführung}

\section{Übersicht über die Applikation}
% Was macht die Applikation? Wie funktioniert sie? Welches Problem löst sie/welchen Zweck hat sie?
Die Applikation Simpsons-Game verfiziert anhand gezielter Fragen an den User, welchem fiktiven Charakter aus einer Auswahl an Charakteren der TV-Serie 'The Simpsons' er am ehesten entspricht. Danach wird eine Übersicht ausgegeben wo der Charakter wohnen und arbeiten wird. 
\section{Start der Applikation}
% Wie startet man die Applikation? Welche Voraussetzungen werden benötigt? Schritt-für-Schritt-Anleitung
Zum Start der Applikation sind folgende Vorrausetzungen notwending:
\begin{itemize}
    \item Das \ac{JDK} um den Code kompilieren und auszuführen zu können.
    \item Ein \ac{IDE} um die Ausführung des Codes komfortabler zu gestalten.
\end{itemize}
Um die Applikation zu starten sollte der Code innerhalb einer \ac{IDE} der Wahl geöffnet werden. Danach muss die Java Klasse 'SimpsonsTerminal' im Ordner 'SimpsonsGame/src/main/java/de/dhbw/ase/simpsons/plugin' ausgeführt werden. Alle Interaktionen der Applikation mit dem User erfolgen anhand einer textbasierten Ausgabe über das Terminal der \ac{IDE}.


\section{Testen der Applikation}
% Wie testet man die Applikation? Welche Voraussetzungen werden benötigt? Schritt-für-Schritt-Anleitung
TODO: Nach den UnitTests hier die Ausführung beschreiben.
