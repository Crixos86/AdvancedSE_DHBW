% ------------------------------------------------------------
% LaTeX Template für die DHBW zum Schnellstart!
% Original: https://github.wdf.sap.corp/vtgermany/LaTeX-Template-DHBW
% ------------------------------------------------------------
% ---- Präambel mit Angaben zum Dokument
\input{Inhalt/00_Latex/praeambel}

% ---- Elektronische Version oder Gedruckte Version?
% ---- Unterschied: Die elektronische Version enthält keinen Platzhalter für die Unterschrift
\usepackage{ifthen}
\newboolean{e-Abgabe}
\setboolean{e-Abgabe}{false}    % false=gedruckte Fassung

% ---- Persönlichen Daten:
\newcommand{\titel}{Das Simpsons-Quiz: Finde deinen Zwilling in Springfield!}
\newcommand{\titelheader}{Simpsons-Quiz}
\newcommand{\arbeit}{Programmentwurf}
\newcommand{\studiengang}{Informatik}
\newcommand{\studienjahr}{2015}
\newcommand{\autor}{Dominik Veith}
\newcommand{\autorReverse}{Veith, Dominik}
\newcommand{\verfassungsort}{Karlsruhe}
\newcommand{\matrikelnr}{3352220}
\newcommand{\kurs}{TINF20B2}
\newcommand{\bearbeitungsmonat}{Dezember 2022}
\newcommand{\abgabe}{28. Mai 2023}
\newcommand{\bearbeitungszeitraum}{04.10.2022 - 28.05.2023}
\newcommand{\firmaName}{SAP SE}
\newcommand{\firmaStrasse}{Dietmar-Hopp-Allee 16}
\newcommand{\firmaPlz}{69190 Walldorf, Deutschland}
\newcommand{\betreuerFirma}{B-Vorname B-Nachname}
\newcommand{\betreuerDhbw}{DH-Vorname DH-Nachname}

\input{Inhalt/00_Latex/kopfundFusszeile}

% ---- Hilfreiches
\newcommand{\zB}{z.\,B. }   % "z.B." mit kleinem Leeraum dazwischen (ohne wäre nicht korrekt)
\newcommand{\dash}{d.\,h. }

\newcommand{\code}[1]{\texttt{#1}} % Ist einfacher zu schreiben als ständig \texttt und erlaubt
                                   % Änderungen im Nachhinein, wenn man z.B. Inline-Code anders stylen möchte.

% ---- Silbentrennung (falls LaTeX defaults falsch / nicht gewünscht sind)
\hyphenation{HANA}         % anstatt HA-NA
\hyphenation{Graph-Script} % anstatt GraphS-cript

% ---- Beginn des Dokuments
\begin{document}
\setlength{\parindent}{0pt}              % Keine Paragraphen Einrückung.
                                         % Dafür haben wir den Abstand zwischen den Paragraphen.
\setcounter{secnumdepth}{2}              % Nummerierungstiefe fürs Inhaltsverzeichnis
\setcounter{tocdepth}{1}                 % Tiefe des Inhaltsverzeichnisses. Ggf. so anpassen,
                                         % dass das Verzeichnis auf eine Seite passt.
\sffamily                                % Serifenlose Schrift verwenden.

% ---- Vorspann
% ------ Titelseite
\singlespacing
\include{Inhalt/01_Standard/titelseite}  % Titelseite
\newcounter{savepage}
\pagenumbering{Roman}                    % Römische Seitenzahlen
\onehalfspacing

% ------ Erklärung, Sperrvermerk, Abstact
% \include{Inhalt/01_Standard/erklaerung}
% \include{Inhalt/01_Standard/sperrvermerk}
% \include{Inhalt/02_Abstract/abstract-en}
%\include{Inhalt/02_Abstract/abstract-de}

% ------ Inhaltsverzeichnis
\singlespacing
\tableofcontents

% ------ Verzeichnisse
\renewcommand*{\chapterpagestyle}{plain}
\pagestyle{plain}
% \include{Inhalt/03_Verzeichnisse/formelgroessen}
\include{Inhalt/03_Verzeichnisse/abkuerzungen}
\listoffigures                          % Erzeugen des Abbildungsverzeichnisses 
%\listoftables                           % Erzeugen des Tabellenverzeichnisses
\renewcommand{\lstlistlistingname}{Quellcodeverzeichnis}
\lstlistoflistings                      % Erzeugen des Listenverzeichnisses
\setcounter{savepage}{\value{page}}


% ---- Inhalt der Arbeit
\cleardoublepage
\pagenumbering{arabic}                  % Arabische Seitenzahlen für den Hauptteil
\setlength{\parskip}{0.5\baselineskip}  % Abstand zwischen Absätzen
\rmfamily
\renewcommand*{\chapterpagestyle}{scrheadings}
\pagestyle{scrheadings}
\onehalfspacing
\chapter{Einführung}

\section{Übersicht über die Applikation}
% Was macht die Applikation? Wie funktioniert sie? Welches Problem löst sie/welchen Zweck hat sie?
Die Fernsehserie 'Die Simpsons' ist eine US-amerikanische Zeichentrickserie, die seit 1989 ausgestrahlt wird und mittlerweile über 700 Episoden hat. Die Serie handelt von der Familie Simpson, bestehend aus Homer, Marge, Bart, Lisa und Maggie, sowie zahlreichen Nebenfiguren, die in der fiktiven Stadt Springfield leben. Die Simpsons zeichnet sich durch ihren satirischen und humorvollen Stil aus, der sowohl politische als auch soziale Themen behandelt. Die Serie parodiert oft bekannte Filme, Serien, Persönlichkeiten und Institutionen und enthält viele popkulturelle Referenzen. Die Simpsons wurde mit zahlreichen Preisen ausgezeichnet, darunter 34 Emmy Awards, und gilt als eine der erfolgreichsten Fernsehserien aller Zeiten. Die Serie wurde in mehr als 100 Ländern ausgestrahlt und hat eine große Fangemeinde auf der ganzen Welt.\cite{simpsons.2023}
\newline
Die Applikation Simpsons-Quiz ist ein Terminal-basierendes Minispiel. Durch gezielte Fragen, welche User:innen mittels Tastatureingaben beantworten, soll bestimmt werden, durch welchen Charakter des Simpsons Universum er oder sie am ehesten repräsentiert wird. Zusätzlich werden Informationen über den Zielcharakter ausgegeben. Dies umfasst den Wohnort, den Arbeitsplatz, die Art der Fortbewegung und das Lieblingsessen der Simpsons-Figur. Zusätzlich werden individuelle, charakterspezifische Fakten präsentiert. \newline
Die visuelle Ausgabe im Terminal wird durch eine \ac{ASCII}- Repräsentation des verifizierten Charakterbildes unterstützt. Nachdem alle Fragen beantwortet wurden und der oder die User:in seinen Simpsons-Charakter mit Erläuterungen erhalten hat, werden alle Informationen zusätzlich in einer Textdatei abgelegt um sie später noch einmal nachlesen zu können.
\newpage
\section{Start der Applikation}
% Wie startet man die Applikation? Welche Voraussetzungen werden benötigt? Schritt-für-Schritt-Anleitung
Zum Start der Applikation sind folgende Voraussetzungen notwendig:
\begin{itemize}
    \item Das \ac{JDK} um den Code kompilieren und auszuführen zu können.
    \item Ein \ac{IDE} um die Ausführung des Codes komfortabler zu gestalten.
\end{itemize}
Um die Applikation zu starten kann der Code innerhalb einer \ac{IDE} der Wahl geöffnet werden. Danach muss die Java Klasse 'SimpsonsTerminal' im Ordner 'SimpsonsQuiz/plugin-0/src/main/java/com/dhbw/ase/simpsons/plugin/SimpsonsTerminal.java' ausgeführt werden. Alle Interaktionen der Applikation mit dem User erfolgen anhand einer textbasierten Ausgabe über das Terminal der \ac{IDE}. Im Anschluss an die Ausführung der Applikation wird eine Textdatei mit dem Namen 'YourCharacter.txt' erstellt. Diese enthält alle Informationen über den Simpsons-Charakter, welcher durch die Antworten des Users bestimmt wurde und im Terminal zu sehen waren. 

\section{Testen der Applikation}
% Wie testet man die Applikation? Welche Voraussetzungen werden benötigt? Schritt-für-Schritt-Anleitung
Um die Applikation zu testen gelten folgende Voraussetzungen:
\begin{itemize}
    \item Maven ist installiert
    \item Optional: Die \ac{IDE} IntelliJ ist installiert
\end{itemize}
Zum Start des Test ist sicherzustellen, dass der Code in der \ac{IDE} der Wahl geöffnet ist. Danach können alle Tests-Klassen im Verzeichnis unter:\newline SimpsonsGame/src/test/java/de/dhbw/ase/simpsons gefunden und einzeln ausgeführt werden. \newline
Alternativ kann im Wurzelverzeichnis 'SimpsonsGame' auch die Kommandozeile geöffnet werden und mit dem Befehl 'mvn test' alle Tests ausgeführt werden. \newline

\chapter{Clean Architecture}
\section{Was ist Clean Architecture?}
% allgemeine Beschreibung der Clean Architecture in eigenen Worten
Die Clean Architecture ist ein Architekturmuster, das sich auf die Trennung von Verantwortlichkeiten und die Abhängigkeiten zwischen den Schichten konzentriert. Die Architektur besteht aus mehreren Schichten, die von innen nach außen, wie in Abbildung \ref{fig:cleanArchitecture} zu sehen, angeordnet sind:

\begin{figure}[h]
    \centering
    \includegraphics[width=0.8\textwidth]{Bilder/CA.png}
    \caption{Clean Architecture \cite{clean.2023}}
    \label{fig:cleanArchitecture}
\end{figure}

\begin{enumerate}
    \item \textbf{Abstraction Code (Schicht 4):} Code, welcher Konzepte, grundlegende Algorithmen und Datenstrukturen implementiert. Dieser Code enthält Domänen-übergreifendes Wissen und wird nur selten berührt.
    \item \textbf{Domain Code (Schicht 3):} Diese Schicht enthält die Kernlogik der Anwendung und wird am seltensten geändert. Hier werden die Geschäftsregeln und -modelle definiert. Die Entitäten der Domäne  werden von der Application Schicht verwendet.
    \item \textbf{Application Code (Schicht 2):} Diese Schicht enthält die Geschäftslogik der Anwendung(Use Cases welche direkt aus Anforderungen resultieren) und koordiniert den Datenfluss mit Hilfe der Entitäten der Domäne. Dabei sind die Regeln der anwendungsspezifischen Logik nicht projektweit gültig.  
    \item \textbf{Adapters (Schicht 1):}
    \item \textbf{Plugins(Schicht 0):}
\end{enumerate}
\section{Analyse der Dependency Rule}
% (1 Klasse, die die Dependency Rule einhält und eine Klasse, die die Dependency Rule verletzt);   jeweils UML der Klasse und Analyse der Abhängigkeiten in beide Richtungen (d.h., von wem hängt die Klasse ab und wer hängt von der Klasse ab) in Bezug auf die Dependency Rule

\subsection{Positiv-Beispiel: Dependency Rule}

\subsection{Negativ-Beispiel: Dependency Rule}

\section{Analyse der Schichten}
% jeweils 1 Klasse zu 2 unterschiedlichen Schichten der Clean-Architecture: jeweils UML der Klasse (ggf. auch zusammenspielenden Klassen), Beschreibung der Aufgabe, Einordnung mit Begründung in die Clean-Architecture

\chapter{SOLID}
SOLID ist ein Akronym für fünf fundamentale Prinzipien der objektorientierten Programmierung:
\begin{enumerate}
    \item Single Responsibility Principle (SRP)
    \item Open Closed Principle (OCP)
    \item Liskov Substitution Principle (LSP)
    \item Interface Segregation Principle (ISP)
    \item Dependency Inversion Principle (DIP)
\end{enumerate}
Sie sollen dazu beitragen, dass Software-Systeme leichter zu verstehen, zu entwickeln, zu testen und zu warten sind. \cite{martin.2017}
\section{Analyse Single-Responsibility-Principle (SRP)}
% jeweils eine Klasse als positives und negatives Beispiel für SRP;  jeweils UML der Klasse und Beschreibung der Aufgabe bzw. der Aufgaben und möglicher Lösungsweg des Negativ-Beispiels (inkl. UML)
Das Prinzip besagt, dass eine Klasse nur eine einzige Verantwortung und Aufgabe haben sollte. Durch die Aufteilung von Aufgaben an mehrere Klassen wird die Klasse lesbarer, wartbarer und hat nur einen Grund geändert zu werden. \cite{martin.2017}
\newpage
\subsection{Positiv-Beispiel}
Ein Beispiel einer Klasse, welche dieses Prinzip umsetzt, ist die Klasse UserBuild aus der Adapter Schicht, wie in Abbildung \ref{fig:UMLUserBuild} zu sehen.
\begin{figure}[ht]
    \centering
    \includegraphics[width=0.4\textwidth]{Bilder/UB.png}
    \caption{Positiv Beispiel SRP der Klasse (UserBuild)}
    \label{fig:UMLUserBuild}
\end{figure}
Ihre primäre Aufgabe ist es, basierend auf der Summe der Character aus den Fragen, die passende Simpsons Figur zu erstellen. 
\subsection{Negativ-Beispiel}
Die Klasse GameTerminal aus der Adapter Schicht hat, wie in Abbildung \ref{fig:UMLGameTerminal} zu sehen, mehrere Aufgaben. Sie ist neben der Erstellung eines Banners und Textes zur Begrüßung des Spieler zusätzlich für die Ausgabe der Fragen und Resultate zuständig.
\begin{figure}[ht]
    \centering
    \includegraphics[width=0.3\textwidth]{Bilder/GT.png}
    \caption{Negativ Beispiel SRP der Klasse (GameTerminal)}
    \label{fig:UMLGameTerminal}
\end{figure}
Diese Aufgaben sollten in zwei verschiedene Klassen aufgeteilt werden, um die Lesbarkeit und Wartbarkeit zu verbessern.
\newpage
\section{Analyse Open-Closed-Principle (OCP)}
% jeweils eine Klasse als positives und negatives Beispiel für OCP;  jeweils UML der Klasse und Analyse mit Begründung, warum das OCP erfüllt/nicht erfüllt wurde – falls erfüllt: warum hier sinnvoll/welches Problem gab es? Falls nicht erfüllt: wie könnte man es lösen (inkl. UML)?
Ein Prinzip, welches besagt, dass Klassen offen für Erweiterungen, aber geschlossen für Änderungen sein sollten. Das bedeutet, dass Änderungen an einer Klasse vermieden werden, sobald sich die Bedingungen ändern. \cite{martin.2017}
\subsection{Positiv-Beispiel}
Im Quiz finden sich 10 Figuren aus dem Simpsons Universum, welche alle aus der Klasse SimpsonsCharacter, wie in Abbildung \ref{fig:UMLSimpsonsCharacter} zu sehen, hervorgehen.

\begin{figure}[ht]
    \centering
    \includegraphics[width=0.3\textwidth]{Bilder/SC.png}
    \caption{Positiv Beispiel OCP der Klasse (SimpsonsCharacter)}
    \label{fig:UMLSimpsonsCharacter}
\end{figure}
Egal wie viele Figuren durch beispielsweise eine Erweiterung noch hinzugefügt werden, die Klasse SimpsonsCharacter muss nicht verändert werden, da sie nur die Grundfunktionen der Figuren enthält.
\subsection{Negativ-Beispiel}
Sollte durch eine Erweiterung eine neuer Figur implementiert werden, wäre die Klasse UserBuild aus Abbildung \ref{fig:UMLUserBuild} ein Verstoss gegen das OCP, da sie bei jeder neuen Figur angepasst werden müsste. Dies ließe sich durch die Auslagerung der Figuren in ein Interface lösen, welches dann als Erweiterung in die Klasse implementiert werden kann.
\section{Analyse Interface-Segreggation- (ISP)}
% jeweils eine Klasse als positives und negatives Beispiel für entweder LSP oder ISP oder DIP);  jeweils UML der Klasse und Begründung, warum man hier das Prinzip erfüllt/nicht erfüllt wird
Das Prinzip besagt, dass eine Klasse nicht von Methoden abhängig sein sollte, die sie nicht benötigt. Durch die Aufteilung von Schnittstellen in kleinere, spezifischere Schnittstellen können unnötige Abhängigkeiten vermieden werden.
\subsection{Positiv-Beispiel}
Ein Beispiel, welche das Prinzip erfüllt, ist das Interface CharacterAction, wie in Abbildung \ref{fig:UMLCharacterAction} zu sehen. Es enthält nur die Methoden, welche für die Figuren benötigt werden. Information über beispielsweise die Features eines Arbeitsplatzes der Figur sind in ein separates Interface ausgelagert worden.
\begin{figure}[ht]
    \centering
    \includegraphics[width=0.3\textwidth]{Bilder/CI.png}
    \caption{Positiv Beispiel ISP der Klasse (CharacterAction)}
    \label{fig:UMLCharacterAction}
\end{figure}

\subsection{Negativ-Beispiel}
Um das genannte Beispiel ins Negative zu kehren, wäre lediglich ein Erweitern des Interfaces um die Methoden aus dem Interface WorkplaceFeatures, wie in Abbildung \ref{fig:UMLWorkplaceFeatures} zu sehen, möglich. Dies würde als Konsequenz bedeuten, dass die Klasse SimpsonsCharacter zusätzlich die Methoden für die Arbeitsplätze verarbeiten müsste und somit unnötige Abhängigkeiten entstehen würden.
\begin{figure}[ht]
    \centering
    \includegraphics[width=0.3\textwidth]{Bilder/WF.png}
    \caption{Negativ Beispiel ISP der Klasse (WorkplaceFeatures)}
    \label{fig:UMLWorkplaceFeatures}
\end{figure}
\chapter{Weitere Prinzipien}
General Responsibility Assignment Software Patterns (GRASP) sind eine Sammlung von Prinzipien, welche in der Softwareentwicklung verwendet werden. In diesem Kapitel werden die Prinzipien Geringe Kopplung und Hohe Kohäsion vorgestellt. Außerdem wird das Prinzip Don't Repeat Yourself (DRY) vorgestellt.
\section{Analyse GRASP: Geringe Kopplung}
% jeweils eine bis jetzt noch nicht behandelte Klasse als positives und negatives Beispiel geringer Kopplung; jeweils UML Diagramm mit zusammenspielenden Klassen, Aufgabenbeschreibung und Begründung für die Umsetzung der geringen Kopplung bzw. Beschreibung, wie die Kopplung aufgelöst werden kann
Low Coupling bezeichnet eine geringe Kopplung zwischen Klassen. Eine geringe Kopplung zwischen Klassen bedeutet, dass eine Klasse nur auf wenige andere Klassen angewiesen ist. Eine Klasse, die auf viele andere Klassen angewiesen ist, ist schwieriger zu warten und zu erweitern. Außerdem ist es schwieriger, die Funktionalität einer Klasse zu testen, wenn diese auf viele andere Klassen angewiesen ist. Zusätzlich lässt sie sich einfacher tauschen oder ersetzen. 
\newpage
\subsection{Positiv-Beispiel}
Ein Beispiel, bei dem das Prinzip der Geringen Kopplung umgesetzt wurde ist die Klasse FlandersHome als Beispiel für die verschiedenen Klassen aus dem Package Homes, welche die Wohnungen der Simpsons Figuren repräsentiert. Wie in Abbildung \ref{fig:flandersHome} zu sehen ist, ist die Klasse FlandersHome nur von der Klasse Home abhängig. Zusätzlich implementiert sie notwendige Methoden des HomeFeature Interfaces. Gibt es beispielsweise eine Änderung am Heim von Rektor Skinner, betrifft es diese Klasse nicht.  
\begin{figure}[ht]
    \centering
    \includegraphics[width=0.3\textwidth]{Bilder/homeI.png}
    \caption{UML Klassendiagramm (FlandersHome) im Zusammenspiel mit Interface}
    \label{fig:flandersHome}
\end{figure}

\subsection{Negativ-Beispiel}
Führt man das zuvor genannte Beispiel eine Ebene nach oben, wird allerdings deutlich, dass die Klasse QuestionManager von jeder Klasse, welche eine der Simpsons Figuren repräsentiert, abhängig ist und somit eine hohe Kopplung aufweist. Abbildung \ref{fig:hoheK} zeigt die Klasse QuestionManager im Zusammenspiel mit den Klassen aus dem Package der Charaktere. 
\begin{figure}[ht]
    \centering
    \includegraphics[width=0.8\textwidth]{Bilder/hoheK.png}
    \caption{UML Klassendiagramm (QuestionManager) in Abhängigkeit der Charaktere}
    \label{fig:hoheK}
\end{figure}

\section{Analyse GRASP: Hohe Kohäsion}
% eine Klasse als positives Beispiel hoher Kohäsion; UML Diagramm und Begründung, warum die Kohäsion hoch ist
Hohe Kohäsion ist wichtig, da Objekte so organisiert werden sollten, dass die Methoden und Attribute zusammengehören und eine klar definierte Aufgabe erfüllen. Zusätzlich erhöht es wesentlich die Überschaubarkeit und des Codes. Abbildung \ref{fig:highK} zeigt die Klassen der einzelnen Figuren im Zusammenhang mit ihren Arbeitsstätten und Wohnorte.
\begin{figure}[ht]
    \centering
    \includegraphics[width=0.8\textwidth]{Bilder/highK.png}
    \caption{UML Klassendiagramm (Charaktere im Zusammenspiel mit Arbeitsstätte und Zuhause)}
    \label{fig:highK}
\end{figure}
\newpage

\section{Don't Repeat Yourself (DRY)}
% ein Commit angeben, bei dem duplizierter Code/duplizierte Logik aufgelöst wurde; Code-Beispiele (vorher/nachher); begründen und Auswirkung beschreiben
DRY ist ein Akronym für Don't Repeat Yourself und ein weiteres Prinzip der Softwareentwicklung. Es bedeutet, dass jede Funktionalität oder Information in einem Programm nur einmal definiert werden sollte um Redundanzen zu vermeiden. So kann beispielsweise Code ausgelagert werden um ihn wiederverwenden zu können ohne ihn zu duplizieren. Außerdem wird die Wartbarkeit des Codes erhöht, da Änderungen nur an einer Stelle vorgenommen werden müssen. So wurde beispielsweise, wie in Listing \ref{code:DRY} zu sehen ist, der Code für die Ausgabe der Charakter Bilder in die SimpsonsCharacter Klasse ausgelagert und kann so direkt von den jeweiligen Klassen aufgerufen werden.
\lstinputlisting[
    label={code:DRY},  % Label; genutzt für Referenzen auf dieses Code-Beispiel
	caption={DRY Prinzip bei Ausgabe der Charakter Bilder},  % Caption; genutzt für Referenzen auf dieses Code-Beispiel
	captionpos=b,               % Position, für die Caption:  t(op) oder b(ottom)
	style=EigenerJavaStyle,     % Eigener Style der vor dem Dokument festgelegt wurde
	firstline=1,                % Zeilennummer im Dokument welche als erste angezeigt wird
	lastline=10     
]{Quellcode/Picture.java}
\chapter{Unit Tests}
% Beschreibung der Tests
Unit Tests haben die Aufgabe einzelne Einheiten von Code auf Funktionalität zu überprüfen. Dabei kann eine Einheit eine einzelne Methode, eine Klasse oder ein Modul sein. Zweck ist die Sicherstellung dass jede Einheit der Software wie erwartet funktioniert und dass Änderungen an einer Einheit keine unerwarteten Auswirkungen auf andere Teile der Software haben. Darüber hinaus helfen Unit-Tests auch dabei, die Qualität und Zuverlässigkeit der Software zu verbessern, da sie sicherstellen, dass jeder Teil des Codes wie erwartet funktioniert und dass Fehler vermieden werden. Unit-Tests tragen somit dazu bei, dass die Software insgesamt stabiler und robuster wird.
\section{10 Unit Tests}
%Nennung von 10 Unit-Tests und Beschreibung, was getestet wird
\begin{enumerate}
    \item UserBuildTest.testPerformActionBasedOnAnswers() Testet ob für jeden Simpsons Charakter eine Aktion bereitgestellt wird.
    \item ApuTest.testIntroduce() Testet ob die Methode introduce() den richtigen String zurückgibt. Die selben Tests wurden für die Simpsons Charaktere Bart, Homer, Marge, Lisa, Maggie, ComicBookGuy, Ned, Skinner und Nelson durchgeführt.
    \item SimpsonsCharacterTest.testFavoriteFood() Testet ob abhängig vom Charakter der Simpsons das spezifische Lieblingsessen zurückgegeben wird.
    \item SimpsonsCharacterTest.testPersonalTransport() Testet ob abhängig vom Charakter der Simpsons das spezifische Transportmittel zurückgegeben wird.
    \item ConsumerGoodsTest.testToString() Testet ob abhängig vom jeweiligen Enum der richtige String zum Lieblingsessen zurückgegeben wird.
    \item PersonalTransportTest.testDisplayName() Testet ob abhängig vom jeweiligen Enum der richtige String zum Transportmittel zurückgegeben wird.
    \item WorkplacesTest.testGettersAndSetters() Testet ob die Getter und Setter der Klasse Workplaces funktionieren.
\end{enumerate}

\section{ATRIP: Automatic}
%Begründung/Erläuterung, wie ‘Automatic’ realisiert wurde
Test sollten nach Möglichkeit automatisiert ablaufen. Des Weiteren sollten auch die Ergebnisse eines Tests auf ihren positiven oder negativen Ausgang geprüft werden. In diesem Projekt wird dies mit dem Surfire Plugin von Maven realisiert, welches, wie in Abbildung \ref{fig:surefire} zu sehen, alle Tests auf einmal abruft und angibt, ob ein Test fehlgeschlagen ist oder nicht. Abgerufen wird dieser Test mit dem Befehl 'mvn test' im Verzeichnis des Projekts über das Terminal.
\begin{figure}[ht]
    \centering
    \includegraphics[width=0.8\textwidth]{Bilder/tests.png}
    \caption{Maven Surfire Plugin}
    \label{fig:surefire}
\end{figure}
\newpage

\section{ATRIP: Thorough}
% jeweils 1 positives und negatives Beispiel zu ‘Thorough’; jeweils Code-Beispiel, Analyse und Begründung, was gründlich/nicht gründlich ist
Thorough bezieht sich auf die Gründlichkeit eines Tests, insbesondere auf die vollständige Prüfung des gesamten Codes um sicherzustellen, dass alle Aspekte der Funktionalität geprüft und alle Szenarien abgedeckt wurden. In diesem Projekt wurde dies durch die Erstellung von Unit Tests für jeden einzelnen Simpsons Charakter sichergestellt, wie in listing \ref{code:apuTest} zu sehen ist. Die Tests prüfen, ob die Methode introduce() den richtigen String zurückgibt. Die selben Tests wurden für die Simpsons Charaktere Bart, Homer, Marge, Lisa, Maggie, ComicBookGuy, Ned, Skinner und Nelson durchgeführt.

\lstinputlisting[
	label=code:apuTest,    % Label; genutzt für Referenzen auf dieses Code-Beispiel
	caption=Unit Test der Klasse Apu,
	captionpos=b,               % Position, an der die Caption angezeigt wird t(op) oder b(ottom)
	style=EigenerJavaStyle,     % Eigener Style der vor dem Dokument festgelegt wurde
	firstline=1,                % Zeilennummer im Dokument welche als erste angezeigt wird
	lastline=12                 % Letzte Zeile welche ins LaTeX Dokument übernommen wird
]{Quellcode/apu.java}
Negativ gilt an dieser Stelle hervorzuheben, dass nicht alle Methoden der jeweiligen Klassen getestet werden. Nur wenn alle Methoden vollständig durch Unit Tests abgedeckt sind, gilt das Prinzip erfüllt.  
\section{ATRIP: Professional}
% jeweils 1 positives und negatives Beispiel zu ‘Professional’; jeweils Code-Beispiel, Analyse und Begründung, was professionell/nicht professionell ist
Da Tests den selben Qualitätsstandards wie Produktivcode unterliegen, sollte auch hierbei darauf geachtet werden, dass die Tests mit der notwendigen Professionalität erstellt werden. Dies bedeutet, dass die Tests gut lesbar und wartbar sind. Außerdem sollten sie so geschrieben werden, dass sie leicht zu verstehen sind und dass sie sich leicht erweitern lassen. Ein positives Beispiel ist die der Test des User Builds in listing \ref{code:userBuildTest}. Dieser Test prüft, ob für jeden Simpsons Charakter eine Aktion bereitgestellt wird. Dies wird durch die Methode testPerformActionBasedOnAnswers() realisiert. Dabei spielt es keine Rolle wie viele Charakter momentan vorhanden sind, oder ob noch welche hinzugefügt werden. 
\lstinputlisting[
    label=code:userBuildTest,    % Label; genutzt für Referenzen auf dieses Code-Beispiel
    caption=Unit Test der Klasse UserBuild,
    captionpos=b,               % Position, an der die Caption angezeigt wird t(op) oder b(ottom)
    style=EigenerJavaStyle,     % Eigender Style der vor dem Dokument festgelegt wurde
    firstline=1,                % Zeilennummer im Dokument welche als erste angezeigt wird
    lastline=9                 % Letzte Zeile welche ins LaTeX Dokument übernommen wird
]{Quellcode/goodtest.java} 
\newpage

Unnötige Tests wie beispielsweise der Test von Getter und Setter Methoden sind hingegen nicht professionell. Dies ist in listing \ref{code:badtest} zu sehen. Dieser Test prüft, ob die Getter und Setter der Klasse Workplaces funktionieren. Dies ist unnötig, da man keine Tests nur des Testens wegen schreiben sollte.
\lstinputlisting[
    label=code:badtest,    % Label; genutzt für Referenzen auf dieses Code-Beispiel
    caption=Unit Test der Klasse Workplaces,
    captionpos=b,               % Position, an der die Caption angezeigt wird t(op) oder b(ottom)
    style=EigenerJavaStyle,     % Eigender Style der vor dem Dokument festgelegt wurde
    firstline=1,                % Zeilennummer im Dokument welche als erste angezeigt wird
    lastline=23                 % Letzte Zeile welche ins LaTeX Dokument übernommen wird
]{Quellcode/badtest.java}
\newpage

\section{Code Coverage}
% Code Coverage im Projekt analysieren und begründen
Code Coverage ist ein Maß dafür, wie viel Prozent des Quellcodes einer Software durch Tests abgedeckt werden. Eine Code Coverage von 100\% würde bedeuten, dass jeder Teil des Codes durch Tests abgedeckt wurde, während eine  niedrigere Code Coverage darauf hinweist, dass einige Teile des Codes nicht durch Tests überprüft wurden. In diesem Projekt wurde die Code Coverage mit dem JaCoCo Plugin von Maven ermittelt. Dieses Plugin ist in Abbildung \ref{fig:jacoco} zu sehen. Die Code Coverage beträgt 24\% der Klassen und 17\% der Methoden, womit also noch Bedarf an Optimierung besteht.
\begin{figure}[ht]
    \centering
    \includegraphics[width=0.8\textwidth]{Bilder/coverage.png}
    \caption{Maven JaCoCo Plugin}
    \label{fig:jacoco}
\end{figure}
\newpage

\section{Fakes und Mocks}
% Analyse und Begründung des Einsatzes von 2 Fake/Mock-Objekten; zusätzlich jeweils UML Diagramm der Klasse
Mocks sind im Kontext von Unit-Tests Platzhalter-Objekte, die das Verhalten von Abhängigkeiten simulieren, die für den Test nicht verfügbar sind oder unerwünschte Nebenwirkungen haben könnten. Sie helfen, Tests schneller auszuführen, indem sie die Interaktionen der zu testenden Komponente mit ihren Abhängigkeiten imitieren. Mocks können auch zur Überprüfung von Interaktionen verwendet werden, indem sie aufgezeichnete Methodenaufrufe und Parameter speichern und anschließend überprüfen, ob sie den erwarteten Werten entsprechen. In diesem Projekt wird, wie in listing \ref{code:mock} zu sehen, der Unit Test QuestionManagerTest als Mock Klasse verwendet um die Eingabe durch einen Nutzer zu simulieren. 
\lstinputlisting[
    label=code:mock,    % Label; genutzt für Referenzen auf dieses Code-Beispiel
    caption=Mock Klasse für den Question Manager,
    captionpos=b,               % Position, an der die Caption angezeigt wird t(op) oder b(ottom)
    style=EigenerJavaStyle,     % Eigender Style der vor dem Dokument festgelegt wurde
    firstline=1,                % Zeilennummer im Dokument welche als erste angezeigt wird
    lastline=33                 % Letzte Zeile welche ins LaTeX Dokument übernommen wird
]{Quellcode/mock.java}

Die Mock-Klasse enthält eine Test-Methode testAskQuestions (), die eine Instanz von QuestionManager erstellt und die askQuestions() - Methode dieser Instanz testet. Zunächst wird eine zufällige Benutzereingabe erstellt, indem eine StringBuilder-Instanz mit der Größe der charToName-Map der QuestionManager-Klasse initialisiert wird. Die Schleife durchläuft dann jede Zeichenposition und fügt zufällig 'y' für Ja oder 'n' für Nein hinzu, um die Benutzereingabe zu simulieren. Anschließend wird eine InputStream-Instanz erstellt, um die generierte Benutzereingabe zu setzen. Dann wird die askQuestions () - Methode aufgerufen, um die Antwort des Frage-Managers zu erhalten. Schließlich wird geprüft, ob das Ergebnis ein gültiges Zeichen enthält, das in der Liste der erwarteten Zeichen 'validChars' enthalten ist.
\chapter{Domain Driven Design}
Domain Driven Design ist ein Ansatz der Software Entwicklung, bei dem die Domäne und deren Logik im Mittelpunkt steht. Dabei hilft DDD der Komplexität von Software vorzubeugen, indem es den Fokus auf das Verständnis der Domäne legt. \cite{ddd.2003}
\section{Ubiquitous Language}
% 4 Beispiele für die Ubiquitous Language; jeweils Bezeichnung, Bedeutung und kurze Begründung, warum es zur Ubiquitous Language gehört
Ubiquitous Language ist ein zentrales Konzept im Domain-Driven Design und bezieht sich auf eine gemeinsame, konsistente Sprache, die von allen Beteiligten im Projekt verwendet wird, um die Domäne und ihre Anforderungen zu beschreiben. Diese einheitliche Sprache soll Kommunikationsprobleme zwischen Entwicklern, Fachexperten, Stakeholdern und anderen Teammitgliedern vermeiden und ein gemeinsames Verständnis der Domäne fördern. In der Praxis bedeutet dies, dass die Begriffe, die in der Geschäftsdomäne verwendet werden, konsistent in den Diskussionen, Dokumentationen und im Code selbst verwendet werden sollten. Ubiquitous Language wird im gesamten Entwicklungsprozess eingesetzt, von der Anforderungsanalyse über das Design bis hin zur Implementierung.
Im Folgenden wird die Ubiquitous Language des Simpsons Quiz Projekts vorgestellt.
\begin{itemize}
    \item Charakter: Ein Charakter ist eine Person der Simpsons Serie. Dabei ist es egal ob es sich um einen Hauptcharakter oder einen Nebencharakter handelt.
    \item Picture: Ein Picture ist ein Bild, welches einem Charakter der Serie zugeordnet ist.
    \item Workplace: Ein Workplace ist ein Arbeitsplatz, welcher einem Charakter der Serie zugeordnet ist.
    \item Home: Ein Home ist ein Wohnort, welcher einem Charakter der Serie zugeordnet ist.
    \item LuxuryFood: Ein LuxuryFood ist das Lieblingsessen, welches einem Charakter der Serie zugeordnet ist.
    \item Transport: Ein Transport ist das präferierte  Transportmittel, welches einem Charakter der Serie zugeordnet ist.
\end{itemize}
Die Begriffe zählen zur Ubiquitous Language, da sie essentiell für das Verständnis des Simpsons Quiz sind und die Hauptmerkmale des Aufbaus der jeweiligen Figur im Rahmen des Quiz darstellen. Dabei werden sie bei der Implementierung als Gruppierung der einzelnen Attribute verwendet.
\section{Entities}
% UML, Beschreibung und Begründung des Einsatzes einer Entity; falls keine Entity vorhanden: ausführliche Begründung, warum es keines geben kann/hier nicht sinnvoll ist
Entitäten repräsentieren Objekte, die innerhalb einer Geschäftsdomäne eine eindeutige Identität besitzen. Im Gegensatz zu Value Objects, die nur durch ihre Attribute definiert sind, haben Entitäten eine Identität, die unabhängig von ihren Eigenschaften ist. Das bedeutet, dass selbst wenn sich der Zustand einer Entität im Laufe der Zeit ändert, ihre Identität konstant bleibt. Beispiele für Entitäten können Kunden, Produkte, Bestellungen oder Mitarbeiter sein. In all diesen Fällen ist die Identität des Objekts entscheidend, um es von anderen Objekten desselben Typs unterscheiden zu können. Im Simpsons Quiz stellen Charakter, Workplace, Home, LuxuryFood und Transport Entitäten dar.
\newpage

\section{Value Objects}
% UML, Beschreibung und Begründung des Einsatzes eines Value Objects; falls kein Value Object vorhanden: ausführliche Begründung, warum es keines geben kann/hier nicht sinnvoll ist
Value Objects sind ein wichtiger Bestandteil von Domain-Driven Design (DDD) und repräsentieren Objekte, die keine eigene Identität haben und ausschließlich durch ihre Eigenschaften oder Attribute definiert sind. Im Gegensatz zu Entitäten, die eine eindeutige Identität besitzen und deren Zustand sich im Laufe der Zeit ändern kann, sind Value
Objects unveränderlich und können bei Bedarf einfach ersetzt werden. Im Simpsons Quiz stellen die einzelne Charaktere des Spiels Value Objects dar. Sie sind anhand ihrer Attribute wie Arbeitsplätze, Vorlieben oder Zitaten einzigartig und können nur durch einen anderen Charakter ersetzt, aber nicht verändert werden.
\begin{figure}[ht]
    \centering
    \includegraphics[width=0.9\textwidth]{Bilder/charakter.png}
    \caption{UML Diagramm für Value Objects}
    \label{fig:ValueObject}
\end{figure}

\section{Repositories}

Repositories vermitteln zwischen der Domäne und dem Modell und stellen Methoden bereit um Aggregates aus dem Persistenzspeicher zu lesen oder zu speichern. Im Simpsons Quiz werden keine Repositories verwendet, da die Abfrage des Aggregates bereits alle Informationen liefert und somit eine weitere Abstraktionsebene überflüssig macht. 

\section{Aggregates}
% UML, Beschreibung und Begründung des Einsatzes eines Aggregates; falls kein Aggregate vorhanden: ausführliche Begründung, warum es keines geben kann/hier nicht sinnvoll ist
Aggregates beziehen sich auf eine Gruppe von zusammenhängenden Entitäten und Value Objects, die eine konsistente Geschäftseinheit bilden. Dabei ist genau geregelt, welche Entitäten und Value Objects zu einem Aggregate gehören und welche nicht, um die Abhängigkeiten abzubilden. \newline
Im Simpsons Quiz gibt es das Aggregate UserBuild, welches aus den Entitäten Charakter, Workplace, Home, LuxuryFood und Transport besteht und mit den Value Objects die Spiel Figur zusammenstellt.

\chapter{Refactoring}

\section{Code Smells}
%[jeweils 1 Code-Beispiel zu 2 Code Smells aus der Vorlesung; jeweils Code-Beispiel und einen möglichen Lösungsweg bzw. den genommen Lösungsweg beschreiben (inkl. (Pseudo-)Code)
Code Smells sind Anzeichen oder Muster in der Software, die auf mögliche Probleme oder schlechte Praktiken im Code hindeuten. Sie sind nicht unbedingt Fehler oder Bugs, können jedoch die Wartbarkeit, Lesbarkeit und Qualität des Codes beeinträchtigen. In diesem Kapitel werden zwei Code Smells im Projekt vorgestellt behoben.
\subsection{Code Smell: Duplicated Code}
Das Vorhandensein von ähnlichem oder identischem Code an mehreren Stellen im Projekt kann auf schlechte Modularisierung oder mangelnde Wiederverwendbarkeit hindeuten. Jeder Simpsons Charakter hat neben seinen Attributen und Methoden, welche sein Zuhause oder Transportmittel beschreiben, auch eine Methode welche nachdem der Charakter ausgewählt wurde, ein entsprechendes Bild über das Terminal ausgibt. Dieser Code ist in allen Charakterklassen vorhanden und wird in jeder Klasse aufgerufen. Dieser Code ist also mehrfach vorhanden und kann durch Auslagern in eine Superklasse gelöst werden. Wie in listing \ref{code:Picture} zu sehen wurde die Methode in die Superklasse SimpsonsCharacter ausgelagert und in den jeweiligen Subklassen nur noch aufgerufen.
\lstinputlisting[
    label=code:Picture,    % Label; genutzt für Referenzen auf dieses Code-Beispiel
    caption=Auslagern der Methode printPicture() in die Superklasse SimpsonsCharacter, % Caption; genutzt für Referenzen auf dieses Code-Beispiel
    captionpos=b,               % Position, an der die Caption angezeigt wird t(op) oder b(ottom)
    style=EigenerJavaStyle,     % Eigender Style der vor dem Dokument festgelegt wurde
    firstline=1,                % Zeilennummer im Dokument welche als erste angezeigt wird
    lastline=10                 % Letzte Zeile welche ins LaTeX Dokument übernommen wird
]{Quellcode/printpicture.java} 


\subsection{Code Smell: Large Class}
Eine Large Class ist eine Klasse, die zu viele Methoden oder eine große Anzahl von Codezeilen hat. Dies kann ein Anzeichen dafür sein, dass die Klasse zu viele Verantwortlichkeiten trägt. Eine große Klasse kann es schwierig machen, den Code zu verstehen, zu warten und zu erweitern. Um dieses Problem zu lösen, sollte die Klasse in kleinere, fokussierte Klassen aufgeteilt werden, die jeweils eine einzige Verantwortung haben. Im Simpsons Quiz gibt es für jeden Charakter bereits eine separate Klasse. Allerdings war bis vor dem Refactoring die Bilder der Charakter ebenfalls mit in der Klasse gespeichert. Das Auslagern in extra Klassen sorgt für eine deutliche Reduktion der Code Zeilen der Charakter Klassen. Im Folgenden Commit sind die Änderungen zu sehen:
\url{https://github.com/Crixos86/AdvancedSE_DHBW/commit/2f4d1b47780bfbeddbe100c476d0e97a3478f384}
\lstinputlisting[
    label=code:picline,    % Label; genutzt für Referenzen auf dieses Code-Beispiel
    caption=Lösung der large class, % Caption; genutzt für Referenzen auf dieses Code-Beispiel
    captionpos=b,               % Position, an der die Caption angezeigt wird t(op) oder b(ottom)
    style=EigenerJavaStyle,     % Eigender Style der vor dem Dokument festgelegt wurde
    firstline=1,                % Zeilennummer im Dokument welche als erste angezeigt wird
    lastline=4                 % Letzte Zeile welche ins LaTeX Dokument übernommen wird
]{Quellcode/picline.java}
\newpage
\section{2 Refactorings}
% 2 unterschiedliche Refactorings aus der Vorlesung anwenden, begründen, sowie UML vorher/nachher liefern; jeweils auf die Commits verweisen
Refactorings sind strukturierte Änderungen am Code, die darauf abzielen, die interne Struktur und Qualität des Codes zu verbessern, ohne das äußere Verhalten oder die Funktionalität der Software zu ändern. Das Hauptziel von Refactoring ist es, den Code lesbarer, wartbarer und besser verständlich zu machen, was die Produktivität der
Entwickler erhöht und die Wahrscheinlichkeit von Fehlern oder Bugs reduziert.
\subsection{Switch Statements}
Im Simpsons Quiz wird je nach Antworten des Spielers ein anderer Charakter der Serie ausgewählt. Dies wurde initial durch ein Switch Statement realisiert was aber nicht sehr gut wartbar ist und auch in Teilen gegen das Open Closed Principle verstößt. Der Code wurde, wie in listing \ref{code:Switch}zu sehen, entsprechend angepasst.
\lstinputlisting[
    label=code:Switch,    % Label; genutzt für Referenzen auf dieses Code-Beispiel
    caption=Refactoring des Switch Statements, % Caption; genutzt für Referenzen auf dieses Code-Beispiel
    captionpos=b,               % Position, an der die Caption angezeigt wird t(op) oder b(ottom)
    style=EigenerJavaStyle,     % Eigender Style der vor dem Dokument festgelegt wurde
    firstline=1,                % Zeilennummer im Dokument welche als erste angezeigt wird
    lastline=16                 % Letzte Zeile welche ins LaTeX Dokument übernommen wird
]{Quellcode/userbuild.java}

Die Verwendung der Runnable Interfaces und einer Map mit Lambdas bieten gegenüber dem Switch Statement einige Vorteile:
\begin{itemize}
    \item Lesbarkeit: Die Verwendung einer Map und Lambdas bietet eine klarere und leichter verständliche Struktur, da sie die Logik zur Ausführung der Aktionen auf einer höheren Abstraktionsebene kapselt.
    \item Erweiterungen: Es ist einfacher, der Map neue Aktionen hinzuzufügen oder vorhandene Aktionen zu ändern, ohne den gesamten Code umzuschreiben. Im Gegensatz dazu erfordert ein Switch-Statement oft eine umfangreichere Änderung des Codes, um neue Fälle hinzuzufügen oder bestehende zu ändern.
    \item Open/Closed Principle: Durch die Verwendung einer Map und Lambdas folgt der Code eher dem Open/Closed Principle, das besagt, dass Software-Einheiten (Klassen, Module, Funktionen usw.) für Erweiterungen offen, aber für Modifikationen geschlossen sein sollten. Hier kann die Implementierung leicht erweitert werden, ohne dass der bestehende Code geändert werden muss.
    \item Effizienz: Da die Map auf Hash-basierten Schlüsseln arbeitet, ist der Zugriff auf die zugehörige Aktion in der Regel schneller als ein Switch-Statement, insbesondere wenn es eine große Anzahl von Fällen gibt.
    \item Flexibilität: Das Runnable-Interface ermöglicht es, Aktionen sowohl synchron als auch asynchron auszuführen. Wenn beispielsweise eine Aktion in einem neuen Thread ausgeführt wird, kann ein Runnable-Objekt an den Thread-Konstruktor übergeben werden.
\end{itemize}
\newpage
\subsection{Polymorphismus}
Polymorphismus ist ein grundlegendes Konzept der objektorientierten Programmierung, das es ermöglicht, verschiedene Objekte durch eine gemeinsame Schnittstelle oder Basisklasse zu behandeln. Polymorphismus kann als Refactoring-Technik verwendet werden, um den Code sauberer, modularer und leichter wartbar zu gestalten. Mit Einführung der SimpsonsCharacter Klasse wurden gemeinsame Eigenschaften und Methoden der Charaktere ausgelagert. Im folgenden Commit ist das Refactoring zu sehen: \url{https://github.com/Crixos86/AdvancedSE_DHBW/commit/7a1e0367ca782b108682dd7c3f19ae087445ccfa} \newline
Folgende Vorteile wurden durch die Einführung der Superklasse erreicht:
\begin{itemize}
    \item Bessere Abstraktion: Polymorphismus ermöglicht es, eine gemeinsame Schnittstelle oder Basisklasse für unterschiedliche Verhaltensweisen oder Implementierungen zu definieren. Dies führt zu einer klareren Abstraktion und kapselt die unterschiedlichen Implementierungen besser.
    \item Erhöhte Wiederverwendbarkeit: Durch die Verwendung von Polymorphismus kann Code, der auf der gemeinsamen Schnittstelle oder Basisklasse basiert, wiederverwendet werden. Dies reduziert die Code-Redundanz und verbessert die Modularität.
    \item Einfachere Erweiterbarkeit: Mit Polymorphismus können neue Verhaltensweisen oder Implementierungen hinzufügt werden, indem neue Klassen erstellt werden, die die gemeinsame Schnittstelle oder Basisklasse erweitern. Dies verbessert die Erweiterbarkeit des Codes und erleichtert das Hinzufügen neuer Funktionen.
\end{itemize}
\chapter{Entwurfsmuster}
% 2 unterschiedliche Entwurfsmuster aus der Vorlesung (oder nach Absprache auch andere) jeweils sinnvoll einsetzen, begründen und UML-Diagramm
\section{Command Pattern}

Das Command-Entwurfsmuster ist ein Verhaltensmuster, das darauf abzielt, die Anfrage zur Ausführung einer Aktion von der eigentlichen Ausführung der Aktion zu entkoppeln. Dies wird erreicht, indem die Aktionen in Objekten gekapselt werden, die als Befehle (Commands) bezeichnet werden. Die Klasse UserBuild, wie in listing \ref{code:command} verwendet das Command-Entwurfsmuster in Kombination mit einer Map, um Aktionen basierend auf den Antworten des Benutzers auszuführen. Das Command-Pattern kapselt eine Anfrage als Objekt, in diesem Fall als Runnable Objekt, sodass es leicht übergeben, gespeichert und ausgeführt werden kann. Im 'UserBuild'-Code werden verschiedene Aktionen in einer Map als Runnable-Objekte gespeichert. Jede Aktion wird durch einen Charakter gekennzeichnet. Die Methode performActionBasedOnAnswers() verwendet diese Map, um die entsprechende Aktion basierend auf der Eingabe (dem meistgewählten Charakter) auszuführen. Das Command-Pattern ermöglicht es, die Aktionen und deren Ausführung zu entkoppeln, sodass sie leicht erweitert oder geändert werden können, ohne die Hauptlogik der Anwendung zu beeinflussen. In diesem Fall ermöglicht das Command-Pattern eine saubere und leicht erweiterbare Implementierung für die Verwaltung von Aktionen, die auf Benutzerantworten basieren.
\newpage
\lstinputlisting[
    label=code:command,    % Label; genutzt für Referenzen auf dieses Code-Beispiel
    caption=Command Pattern der UserBuild Klasse, % Caption; genutzt für Referenzen auf dieses Code-Beispiel
    captionpos=b,               % Position, an der die Caption angezeigt wird t(op) oder b(ottom)
    style=EigenerJavaStyle,     % Eigender Style der vor dem Dokument festgelegt wurde
    firstline=1,                % Zeilennummer im Dokument welche als erste angezeigt wird
    lastline=34                 % Letzte Zeile welche ins LaTeX Dokument übernommen wird
]{Quellcode/command.java}

\section{Strategy Pattern}
Das Strategy-Entwurfsmuster ist ein Verhaltensmuster, das darauf abzielt, eine Familie von austauschbaren Algorithmen zu definieren, die unabhängig von den Clients verwendet werden können, die sie verwenden. Das Strategy-Muster ermöglicht es, den Algorithmus, der von einem Objekt verwendet wird, dynamisch zur Laufzeit zu ändern, ohne das Objekt selbst zu ändern. Die Klasse QuestionManager verwendet, wie in listing \ref{code:strategy} zu sehen, einige Aspekte des Strategy-Musters, indem sie die Fragen und zugehörigen Antworten in einer Map speichert. Dies ermöglicht es, die Fragen und Antworten leicht zu ändern oder zu erweitern, ohne die Hauptlogik der Klasse ändern zu müssen.
In diesem Fall ist es jedoch nicht vollständig angewendet, da die Algorithmen selbst (die Fragen) direkt in der Klasse definiert sind und keine austauschbaren Strategieobjekte verwendet werden.

\lstinputlisting[
    label=code:strategy,    % Label; genutzt für Referenzen auf dieses Code-Beispiel
    caption=Strategy Pattern der QuestionManager Klasse, % Caption; genutzt für Referenzen auf dieses Code-Beispiel
    captionpos=b,               % Position, an der die Caption angezeigt wird t(op) oder b(ottom)
    style=EigenerJavaStyle,     % Eigender Style der vor dem Dokument festgelegt wurde
    firstline=1,                % Zeilennummer im Dokument welche als erste angezeigt wird
    lastline=104                 % Letzte Zeile welche ins LaTeX Dokument übernommen wird
]{Quellcode/question.java}

% ---- Literaturverzeichnis
\cleardoublepage
\renewcommand*{\chapterpagestyle}{plain}
\pagestyle{plain}
\pagenumbering{Roman}                   % Römische Seitenzahlen
\setcounter{page}{\numexpr\value{savepage}+1}
\printbibliography[title=Literaturverzeichnis]

% ---- Anhang
\appendix
%\clearpage
%\pagenumbering{Roman}  % römische Seitenzahlen für Anhang

\newpage
\end{document}
